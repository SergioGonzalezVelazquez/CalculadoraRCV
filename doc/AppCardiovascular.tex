%%%%%%%%%%%%%% 
% Fichero: Principal
% Autor: S. González Velázquez
% Fecha: febrero, 2019
% Descripción: Inteligencia Artificial Aplicada a la Medicina
% Trabajo Teórico de la asignatura Gestión de Sistemas de 
% información
% Curso: 2018/2019
%%%%%%%%%%%%%%

%%%%%%%%%%%%%%
% Preámbulo del documento
%%%%%%%%%%%%%%
\documentclass[ 		% Clase del documento
	11pt,				% Tamaño de letra
	a4paper,			% Tamaño de papel
	oneside,			% Impresión a única cara
	openany,			%Inicia cap. en cualquier página
	final       		% Versión final
]{book}

\usepackage[utf8]{inputenx} % Codificación de entrada
\usepackage[english,spanish,es-tabla,es-noindentfirst]{babel} % Internacionalización

%--- Geometría de las páginas del documento
\usepackage[			% Márgenes del documento
	top=2.5cm,			% Margen superior
	bottom=2.5cm,		% Margen inferior
	inner=3.5cm,		% Margen al interior
	outer=2cm			% Margen al exterior
]{geometry}

%--- Tipografía
\usepackage{textcomp,marvosym,pifont} % Generación de símbolos especiales
\usepackage{ccicons} % Iconos de licencia Creative Commons
\usepackage{amsmath,amsthm,amssymb}	% Mejoras cuando hay matemáticas

%--- Tipografía (Opción 1)
\usepackage[tt=false]{libertine}	% Libertine
\usepackage[libertine]{newtxmath}	% Times
%---


\usepackage[T1]{fontenc}% Codificación de salida    
\usepackage{microtype}	% Mejoras de microtipografía en la obtención de PDF (sólo para pdflatex)

\usepackage{url} 		% Para escritura de URL
\urlstyle{sf}			% Estilo sans serif para URLs

%--- Definición de colores
% Este paquete debe cargarse antes de ctable.
% Revisar la documentación del paquete para ver los nombres de colores predefinidos.
\usepackage[%
	usenames,
	dvipsnames,
	svgnames,
	x11names,
	table
]{xcolor}

%--- Color especial definido para los hiperenlaces
\definecolor{palered}{rgb}{.8,0,0}


%--- Generación de hiperenlaces
\usepackage[pdftex]{hyperref}
\hypersetup{%
	breaklinks,			% Permite que los links ocupen más de una línea
%	hidelinks,			% Oculta el color y borde de los links
% OJO: La opción colorlinks se comenta para evitar un error con el paquete menukeys. 
	colorlinks=true,	% Pone color en los link o un borde
	linkcolor=palered,	% Color de los links
	anchorcolor=palered, 
	citecolor=palered, 
	filecolor=palered, 
	menucolor=palered, 
	urlcolor=palered,
	bookmarksnumbered=true, % Incluye números en bookmarks
% EDITAR: Valores para el PDF.	
	pdftitle={Inteligencia Artificial Aplicada a Medicina}, 	% Título
	pdfauthor={Sergio Gónzalez / Antonio Rubio},	% Autor
	pdfsubject={Gestión de Sistemas de Información},	% Tema
	pdftoolbar=true,	% Muestra la toolbar de Acrobat
	pdfmenubar=true		% Muestra la menubar de Acrobat
}


%--- Paquetes para listas y organización de texto
\usepackage{paralist}	% Mayor control de listas
\usepackage{multicol}	% Texto en varias columnas

%--- Gráficos y tablas
\usepackage{graphicx}	% Inclusión de figuras
\usepackage{subfigure}	% Inclusión de subfiguras
% EDITAR: Si es necesario cambiar el path para los directorios de figuras.
\graphicspath{{./figs/}}% Path de búsqueda de ficheros gráficos
\DeclareGraphicsExtensions{.pdf,.png,.jpg} % Precedencia de extensiones
\usepackage{rotating}	% Giro de cajas (texto, figuras, tablas) (No DVI)
\usepackage{tabularx,booktabs}	% Ajustes para tablas


%--- Paquetes especiales para Informática
\usepackage{listings}	% Inclusión de listados de código

%--- Personalización de títulos de figuras y tablas
\usepackage[%
	margin=10pt,		% Margen
	font=small,			% Tamaño de tipografía
	labelfont=bf,		% Prefijo-Etiqueta en negrita
	format=hang			%
]{caption}
\captionsetup[table]{skip=4pt} 	% Separación del caption en las tablas

%--- Bibliografía: Biblatex con biber.
\usepackage[backend=biber,
			sortcites,
			style=numeric-comp]{biblatex}
\usepackage[autostyle]{csquotes}

% OJO: Línea añadida para eliminar el idioma de las fuentes bibliográficas que no están en el idioma principal del documento.
\AtEveryBibitem{\clearfield{note} \clearlist{language}}

% OJO: Poner siempre fichero con extensión
\addbibresource{./biblatex.bib}


%--- Paquete con personalización local para el TFG (ESI-UCLM)
\usepackage{uclmTFGesi}
% -------------------------
% PAQUETES QUE USA: 
% 	fancyhdr,titlesec,sectsty,tikz
% -------------------------
% LISTA DE COMANDOS PROPORCIONADOS:
% OB-Obligatorio
% OP-Opcional
% RE-Recomendado
%
% Comandos para definir variables con los datos del documento.
%
% OB-\portadaTFG				: Pág. de portada (usa variables definidas)
% OB-\portadillaTFG				: Pág. de portada interna
% OB-\tribunalTFG				: Pág. tribunal
% RE-\dedicado{texto}			: Pág. de dedicatoria con texto
% RE-\creditos{texto}{imagen}	: Pág. de créditos con el texto y la imagen
% OB-\abstract{texto}			: Añade abstract (no definido en clase book)
% OP-\tecla{texto}				: Borde de tecla en torno al texto (sustituir por menukeys)		
% OP-\nodivide[penalty]			: Penaliza la división de palabras. 
%								  Máx. (n=10000) sin arg.
% OP-\nowidowandorphan[penalty]: Penaliza las viudas y huérfanas.
%								(sólo si necesario)
% OP-\nodividenotes[penalty]	: Penaliza la división de notas al pié 
%								entre págs.	(sólo si necesario)	
% OP-\savepagecnt				: Crea contador interno con el nº de pág. actual
% OP-\contpagination			: Recupera el valor de pág. previamente
%								 salvado en el cont. interno.
% RE-\cleanhdfirst				: Elimina la cabecera en la primera 
%								página de capítulo.
%
% -------------------------

\usepackage{verbatim} % comentarios
\usepackage{wallpaper}
\usepackage{color}
\usepackage{wrapfig}


%
% CUERPO DEL DOCUMENTO
%
% -------------------------
\begin{document}
	%PORTADA
	\frontmatter
	% Cambia la numeración de páginas a números romanos y las secciones no están numeradas aunque si aparecen en el índice de contenidos.
	\begin{comment}
	\begin{titlepage}
			\begin{center}
				\includegraphics[width=4cm]{uclm}\vspace{1cm}
				
				{\LARGE \textbf{UNIVERSIDAD DE CASTILLA-LA MANCHA\\[1.5\parskip]
						ESCUELA SUPERIOR DE INFORMÁTICA\\[4cm]}}
						
				{\huge \textbf{Inteligencia Artificial aplicada a la Medicina\\[1.5\parskip]}}	
			\end{center}
	\end{titlepage}
	\end{comment}
	
	\begin{titlepage}
	    \newgeometry{left=0cm,top=0cm,bottom=0cm, right=0cm}
	    \AddToShipoutPicture*{\put(0,0){\includegraphics[scale=1]{portada.pdf}}} % pdf de fondo de portada
	    \noindent
	    \vspace{125mm}
	\end{titlepage}
	
	%--- Resumen en español
	\chapter*{Resumen}
	% PENDIENTE: Resumen	

	\textbf{TERMINAR}

	\addcontentsline{toc}{chapter}{Resumen}
	
	
	% -------------------------
	%
	% ÍNDICES
	%
	% -------------------------
	\pagestyle{fancy} % Estilo de página ajustado por fancyhdr
	
	% EDITAR: Si alguno de los índices no existe, su inclusión se puede comentar.
	\tableofcontents  % Índice general
	
	%--- MAINMATTER
	% Capítulos del documento
	% Salva en un contador interno el nº de páginas actual
	% Debe ir antes de \mainmatter (antes de que se reinicie el cnt page)
	\savepagecnt
	\mainmatter
	% Justo antes del primer capítulo del libro. Activa la numeración con números arábigos y reinicia el contador de páginas.
	
	% OJO: No cambiar de ubicación. Elimina cabecera y pie en pág. inicial de cap.
	\cleanhdfirst
	
	% Reajuste del número de página consecutivo para no reiniciar paginación en Cáp. 1
	%\contpagination % Comentado para reiniciar paginación (pag. 1)
	
	
	
	
	% -------------------------
	%
	% CAPÍTULOS
	%
	% -------------------------
	% Se incluye un fichero para cada capítulo. Se emplea la instrucción \include porque en los libros lo más habitual es que cada cápítulo comience en una nueva página.
	\chapter {Introducción}
\label{cap:Introduccion}

Si algo valoran especialmente en su trabajo cotidiano los médicos de familia es la aproximación a la incertidumbre. Cuando se encuentran a su alcance las pruebas diagnósticas en cada caso y disponen de guías de tratamiento actualizadas, basadas en evidencias y adaptadas a su quehacer cotidiano, experimentan cómo disminuye la incertidumbre propia de su trabajo. 
Esta situación queda destacada con especial relevancia cuando se trata de atender a los pacientes que sufren las enfermedades cardiovasculares. En España, las enfermedades del aparato circulatorio constituyeron la principal causa de muerte para el conjunto de la población, representando el 36\% de todas las defunciones. 

\section{Propuesta de Proyecto}
Según diversos estudios presentados en el Congreso de eSalud, la Inteligencia Artificial puede
ayudar al diagnóstico y tratamiento de enfermedades cardiovasculares, la primera causa de
morbimortalidad y gasto sanitario.

En este sentido, nos gustaría realizar un prototipo de una aplicación para detectar el riesgo de
enfermedad cardiovascular en función del peso, edad, alimentación, frecuencia con que se
práctica deporte y otros factores a estudiar. El programa deberá analizar la información
introducida por el paciente y, comparando con su base de datos, establecer las probabilidades de
que sufra una enfermedad cardiovascular. Es decir, el objetivo de este trabajo es hacer medicina
predictiva identificando signos que preceden a algunas enfermedades cardíacas concretas.

La idea ha sido tomada del desafío social propuesto por Mary Luz Mouronte López para la edición
2019 del programa HackForGood.
\begin{center}
\url{https://hackforgood.net/reto-ufv-1-riesgo-de-enfermedad-cardiovascular/}
\end{center}


\section{Descripción y justificación del trabajo}
En el trabajo teórico de la asignatura Gestión de Sistemas de Información, hemos investigado sobre la Inteligencia Artificial y su aplicación en Medicina. Hablábamos de los Sistemas Expertos como uno de los primeros resultados de la Inteligencia Artificial, porque logran resolver problemas a través del conocimiento, de forma similar a como lo hace un ser humano. Decíamos también, que una de las aplicaciones más importantes de los sistemas expertos tiene lugar en Medicina, donde pueden utilizarse principalmente para diagnóstico médico.

Por otro lado, concluíamos nuestro trabajo haciendo una pequeña reflexión sobre el futuro que nos espera gracias a los avances de la Inteligencia Artificial. En este sentido, proponíamos un cambio de chip que nos conduzca a marcarnos otros objetivos más ambiciosos: hacer \textbf{medicina preventiva} intentando evitar la aparición del mayor número de enfermedades, basándonos en actuaciones y consejos médicos que se pueden anticipar gracias a los avances de la tecnología.

La combinación de Sistemas Expertos con la idea de medicina preventiva constituye la raíz del origen de nuestro proyecto de laboratorio. De manera más concreta, hemos intentado desarrollar una aproximación de Sistema Experto en forma de aplicación Android que se pueda utilizar para prevenir y reducir el número de enfermedades cardiovasculares. Decimos que se trata de una aproximación de Sistema Experto, porque hemos desarrollado un algoritmo (\textbf{motor de inferencia}) que permita relacionar información concreta sobre el estilo de vida de un usuario (\textbf{base de hechos}), con una serie de conocimiento general sobre los factores que influyen en el riesgo de sufrir una enfermedad cardiovascular (\textbf{base de conocimiento}).

Tal y como explicábamos en el trabajo teórico, la base de conocimiento de un SE está formada por todo el conocimiento disponible sobre el campo en el que se desarrolla la aplicación. En nuestro caso particular, este conocimiento lo encontramos en el \textbf{modelo para el cálculo de la probabilidad de riesgo cardiovascular de Framingham}. 

\section{Herramientas utilizadas}

\subsubsection{Entorno de Desarrollo}
Hemos decidido implementar nuestra aplicación en Android haciendo uso del entorno de desarrollo \textbf{Android Studio} y del lenguaje de programación\textbf{ Java}. La principal motivación de esta decisión es que nos gustaría aprender a programar en esta plataforma, pues creemos que es bastante versátil, compatible con una inmensidad de dispositivos y que nos ofrece la posibilidad de desarrollar aplicaciones que estén disponibles para millones de usuarios en todo el mundo. Por tanto, consideramos que es muy recomendable aprender a programar en Android y nos puede resultar de utilidad en el futuro.

Además, otra de las razones que nos conducen a decantarnos por Android es que, al tratarse de una plataforma de código abierto, podemos encontrar una gran cantidad de publicaciones que nos ayudarán a desarrollar nuestro trabajo.


\subsubsection{Repositorio de código}
Hemos utilizado un repositorio de código privado en Github que nos ha permitido de tener un control de las versiones de nuestra práctica, además de ofrecer la posibilidad de trabajar de forma paralela todos los miembros del grupo de trabajo.

\begin{center}
\url{https://github.com/SergioGonzalezVelazquez/ProyectoGSI}
\end{center}

\subsubsection{Base de Datos}
Para añadir persistencia a los datos de nuestra aplicación Android, hemos utilizado un motor ligero de bases de datos de código abierto como es \textbf{SQLite}, una tecnología muy cómoda para los dispositivos móviles. Su simplicidad, rapidez y usabilidad permiten un desarrollo muy amigable. El conector que hemos utilizado para que nos proporcione los mecanismos básicos para la relación entre la aplicación Android y la información, es \textbf{SQLiteOpenHelper}.

\subsubsection{Librerías de terceros}
\begin{itemize}
\item \textbf{HelloCharts}. Librería para implementar gráficos estadísticos compatible con API 8+(Android 2.2).

\begin{center}
\url{https://github.com/lecho/hellocharts-android}
\end{center}

\item \textbf{JustifiedTextView}. Librería que permite justificar texto en una aplicación Android.
\begin{center}
\url{https://github.com/amilcar-sr/JustifiedTextView}
\end{center}


\item \textbf{CircleImageView}. Librería que permite introducir imágenes con forma circular. 

\begin{center}
\url{https://github.com/hdodenhof/CircleImageView}
\end{center}
 


\end{itemize}

	\chapter {Fase I: Análisis}
\label{cap:Fase I. Análisis}

Hasta hace algo más de una década, las recomendaciones clínicas en la prevención cardiovascular iban dirigidas fundamentalmente al manejo independiente de sus factores de riesgo. Con este enfoque, y de forma sistemática, se fueron elaborando guías para el abordaje de cada uno de los factores de riesgo. Sin embargo,  nuestro trabajo se apoya en estudios que toman los factores de riesgo como un conjunto para realizar una estimación precisa de la probabilidad de sufrir una enfermedad cardiovascular. 


\section{Factores de Riesgo cardiovascular}
Los factores de riesgo cardiovascular son condicionantes ligados a estilos de vida que incrementan la probabilidad de padecer o morir por enfermedad vascular en aquellas personas en las que inciden. Se catalogan como tales cuando cumplen unos requisitos que permiten establecer una relación de causa-efecto con respecto a la enfermedad vascular. Se pueden clasificar como se muestra en la siguiente figura. Subrayados en amarillo se muestran los que utiliza el algoritmo de Framimgham, tal y como se explicará más adelante. 

\begin{figure}[htb]
	\centering
	\includegraphics[width=\textwidth]{factores} 
	\caption[Tabla Factores]{Factores de Riesgo Cardiovascular (FRCV) \cite{tagle2007estimacion}
	}
	\label{fig:factores}
\end{figure}

Estos factores de riesgo cardiovasculares, ya sea de forma aislada o como sucede con mucha mayor frecuencia, en combinación, explican la mayoría de casos de enfermedad vascular o de muerte que ocurre en individuos de alto riesgo y una proporción considerable de casos en la población general.


\section{Estimación del Riesgo cardiovascular}
El Riesgo Cardiovascular expresa la probabilidad de sufrir un evento de la enfermedad vascular en un determinado período de tiempo, generalmente 5 o 10 años. La explicación en la práctica de este hecho es que de 100 personas con un mismo porcentaje de riesgo cardiovascular (Ej. RCV de 21\%), de esas 100 personas, 21 desarrollaran una Enfermedad Vascular en los próximos 5 o 10 años.

Para la prevención cardiovascular es necesaria una valoración conjunta de los factores de riesgo mediante la estimación del riesgo cardiovascular del individuo. Esto, permite definir niveles de riesgo con los que el personal sanitario y los propios pacientes pueden valorar cómo se modifica el mismo a medida que se van logrando los objetivos pactados.

Existen diversos métodos para estimar el RCV y ninguno de ellos es perfecto. Sin embargo, se ha comprobado que son las\textbf{ tablas de
Framingham } las que mejor se comportan a la hora de predecir la aparición de eventos cardiovasculares:

\begin{itemize}
\item Tienen \textbf{riesgo cardiovascular alto} los pacientes que reúnen una puntuación de Framingham de 22 o superior, lo que supone una probabilidad de padecer un evento cardiovascular en los próximos 10 años superior al 20\%. También consideraremos en riesgo alto a aquellos que ya han sufrido un episodio cardiovascular o presentan diabetes. 

\item Se califican de \textbf{riesgo cardiovascular moderado} aquellos pacientes con factor de riesgo (hipertensión arterial o tabaquismo) y una puntuación inferior al 22, y consiguientemente una probabilidad de episodio isquémico inferior al 20\% en los próximos 10 años.

\item Son de \textbf{riesgo cardiovascular bajo} los pacientes, independientemente de la edad y el sexo, sin factores de riesgos reconocidos. 
\end{itemize}

El cálculo de valoración al que nos referimos se reproduce en las tablas ~\ref{fig:raCalculo1} y ~\ref{fig:raCalculo2}. Es importante tener en cuenta que existen factores de riesgo cardiovascular no incluidos en la tabla, como el sedentarismo, obesidad, y sobre todo el antecedente familiar aparecido en edad precoz (antes de 55 años en familiares varones y de 65 en mujeres), que deben ser considerados para establecer un plan personalizado de control de riesgo en cada paciente.  
\begin{figure}[htb]
	\centering
	\includegraphics[width=\textwidth]{tablaFir} 
	\caption[Tabla Framighan]{Tablas de cálculos de RCV del estidoo de Framingham (Anderson, 1991) \cite{tagle2007estimacion}
	}
	\label{fig:tablaFir1}
\end{figure}

\begin{figure}[htb]
	\centering
	\includegraphics[width=\textwidth]{tablaFir2} 
	\caption[Predicción Framighan]{Tabla de puntuación y porcentaje de riesgo en los próximos 10 años \cite{tagle2007estimacion}
	}
	\label{fig:tablaFir2}
\end{figure}





\section{Implementando el Algoritmo de Framingham}
Una vez estudiado qué es y cómo se estima el riesgo cardiovascular, el objetivo que nos proponemos en este trabajo es el desarrollo de una aplicación que implemente el algoritmo de Framimgham, de forma que, para la información particular introducida por un sujeto, sea capaz de calcular el porcentaje de riesgo de sufrir una enfermedad cardiovascular en un periodo de 10 años. 

La utilidad práctica del uso de la tabla, y consecuentemente de nuestra aplicación, radica en que permite:
\begin{itemize}
\item Priorizar los cuidados, controles y seguimientos en aquellas personas que presentan un mayor riesgo.

\item Apoyar en las decisiones de tratamiento farmacológico en cuanto a hipolipemiantes, antihipertensivos y antiagregantes.

\item Monitorizar la evolución del RCV

\item Constituir una herramienta educativa y motivacional para el paciente en cuanto a la obtención de objetivos para la reducción de su riesgo.
\end{itemize}




	\chapter {Fase II: Diseño}
\label{cap:Fase II.Diseño}

Medicina y Tecnología han caminado juntas desde el principio de los tiempos: desde el instrumental quirúrgico y su evolución a lo largo de los siglos, pasando por el fonendoscopio, las prótesis, el electrocardiograma, las radiografías o la resonancia magnética. Son muchos los avances tecnológicos que han facilitado la labor de los médicos a la hora de emitir diagnósticos y aplicar tratamientos.

\section{Prototipado de la GUI}
\subsection{Ventana de Inicio}
\begin{figure}[htb]
	\centering
	\includegraphics[width=0.4\textwidth]{GSI-1} 
	\caption[Ventana de Inicio]{Ventana de inicio}
	
	\label{fig:defInicio}
\end{figure}


\subsection{Ventana de Estado}
\begin{figure}[htb]
	\centering
	\includegraphics[width=0.4\textwidth]{GSI-2} 
	\caption[Ventana de Estado]{Ventana de Estado}
	
	\label{fig:defEstado}
\end{figure}

\subsection{Ventana de Perfil}
\begin{figure}[htb]
	\centering
	\includegraphics[width=0.4\textwidth]{GSI-3} 
	\caption[Ventana de Perfil]{Ventana de Perfil}
	
	\label{fig:defPerfil}
\end{figure}

\subsection{Ventana de Cálculo del RCV}
\begin{figure}[htb]
	\centering
	\subfigure[]{
		\includegraphics[width=0.3\textwidth]{GSI-4} 
		\label{fig:raCalculo1}
	}
	\subfigure[]{
		\includegraphics[width=0.3\textwidth]{GSI-5}
		\label{fig:raCalculo2}
	}
		\subfigure[]{
			\includegraphics[width=0.3\textwidth]{GSI-6}
			\label{fig:raCalculo3}
		}
		\subfigure[]{
			\includegraphics[width=0.3\textwidth]{GSI-7}
			\label{fig:raCalculo4}
		}
		\subfigure[]{
			\includegraphics[width=0.3\textwidth]{GSI-8}
			\label{fig:raCalculo5}
		}
		\subfigure[]{
			\includegraphics[width=0.3\textwidth]{GSI-9}
			\label{fig:raCalculo6}
		}
		\subfigure[]{
			\includegraphics[width=0.3\textwidth]{GSI-10}
			\label{fig:raCalculo7}
		}
	\caption{Ventana Cálculo del RCV}
	\label{fig:cirugiaRA}
\end{figure}



\textbf{\textcolor{red}{\huge PENDIENTE}}

	

	% OJO: Añadir para que hiperenlaces de índice salgan bien.
	\refstepcounter{chapter} % Incrementa el contador al nivel indicado (el subnivel se reinicia). Además la referencia se apunta correctamente (necesario para hyperref genere bien los enlaces).
	% -------------------------
	
	%--- BACKMATTER
	\backmatter
	

	% -------------------------
	%
	% BIBLIOGRAFÍA
	%
	% -------------------------
	% OJO: Todas las referencias deben estar citadas en el texto)
	% EDITAR: Comentar línea siguiente
	\nocite{*} % INCLUIDO para ver cómo queda, pero comentar en versión final.
	
	\addcontentsline{toc}{chapter}{\bibname} % Añade la bibliografía al Índice de contenidos.
	

	%---
	% Opción 2: Bibliografía con secciones separadas.
	%---
	\printbibheading
	\printbibliography[heading=subbibliography,nottype=online,title={Fuentes no online}]	
	\printbibliography[heading=subbibliography,type=online,title={Fuentes online}]

	% -------------------------

	% -------------------------
\end{document}

